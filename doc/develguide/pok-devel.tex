%
%                               POK header
%
% The following file is a part of the POK project. Any modification should
% be made according to the POK licence. You CANNOT use this file or a part
% of a file for your own project.
%
% For more information on the POK licence, please see our LICENCE FILE
%
% Please follow the coding guidelines described in doc/CODING_GUIDELINES
%
%                                      Copyright (c) 2007-2024 POK team

\documentclass[oneside]{article}
\usepackage{xspace}
\usepackage{epsfig}
\usepackage{listings}
\usepackage{graphicx}
\usepackage{pslatex}
\usepackage{epsfig}
\usepackage{url}
\usepackage[english]{babel}

\usepackage{rotating}

\newcommand{\Concept}[1]{#1\xspace}
\newcommand{\aadl}{\Concept{AADL}}
\newcommand{\ada}{\Concept{Ada}}
\newcommand{\arinc}{\Concept{ARINC653}}
\newcommand{\scade}{\Concept{SCADE}}
\newcommand{\simulink}{\Concept{Simulink}}
\newcommand{\leon}{\Concept{Leon3}}
\newcommand{\powerpc}{\Concept{PowerPC}}
\newcommand{\qemu}{\Concept{QEMU}}
\newcommand{\pok}{\Concept{POK}}
\newcommand{\macosx}{\Concept{Mac OS X}}
\newcommand{\windows}{\Concept{Windows}}
\newcommand{\linux}{\Concept{Linux}}
\newcommand{\intel}{\textsc{Intel}}
\newcommand{\ocarina}{\textit{Ocarina}\xspace}
\newcommand{\susp}{.\xspace}

\lstdefinelanguage{aadl}
{morekeywords={aadlboolean,aadlinteger,aadlreal,aadlstring,access,all,and,
        annex,applies,binding,bus,calls,classifier,connections,constant,
        data,delta,device,end,enumeration,event,extends,false,features,flow,
        flows,group,implementation,in,inherit,initial,inverse,is,list,memory,
        mode,modes,none,not,of,or,out,package,parameter,path,port,private,
        process,processor,virtual,properties,property,provides,public,range,
        reference,refined,refines,requires,server,set,sink,source,
        subcomponents,subprogram,system,thread,to,true,type,units,value},
morecomment=[l]{--}}

% Layout for listings

\lstset{language=aadl,
        basicstyle=\scriptsize\sffamily,
        aboveskip=.1cm, % \smallskipamount, % \bigskipamount,
        belowskip=.1cm, % \smallskipamount, % \bigskipamount,
        abovecaptionskip=.1cm, % \smallskipamount, % \medskipamount,
        belowcaptionskip=.1cm, % \smallskipamount, % \bigskipamount,
        xleftmargin=.0cm,
        numbersep=-10pt,
        captionpos=b,
        numbers=right,
        tabsize=3}



\newcommand{\onehugefig}[3]{%
  \begin{figure}[htbp]
    \centerline{\epsfig{file=#1.pdf,width=.60\textwidth}}
    \caption{#2}
    \label{#3}
  \end{figure}
}

\newcommand{\onefullfig}[3]{%
  \begin{figure}[htbp]
    \centerline{\epsfig{file=#1.pdf,width=.45\textwidth}}
    \caption{#2}
    \label{#3}
  \end{figure}
}



\newcommand{\onemedfig}[3]{%
  \begin{figure}[htbp]
    \centerline{\epsfig{file=#1.pdf,width=.35\textwidth}}
    \caption{#2}
    \label{#3}
  \end{figure}
}


\begin{document}

\title{POK Developper Guide}

\author{POK Team}

\date{\today}


\maketitle

\tableofcontents

%\listoffigures
\section{About this manual}
This manual provides information about the development of POK. It indicates
coding rules and naming convention so that everybody could improve POK by
modifying its source code.

   \subsection{About POK}
   POK is a free operating system compliant with the ARINC653 and MILS
   standards. It provides time and space partitioning to isolate software in
   partitions. POK is released under the BSD licence and thus, could be modified
   and used for commercial as well as non-commercial use. To have more
   information about the licence of the projet, see POK
   website\footnote{http://pok.gunnm.org}.

\section{Coding guidelines}
There are our coding guidelines:
\begin{enumerate}
    \item
      Prefix for types: \texttt{pok\_}
   \item
      Prefix for functions : \texttt{pok\_}
   \item
      Prefix for maccro: \texttt{POK\_} but not for conditional compiling:
      \begin{enumerate}
         \item
            When the code needs a functionnality, we define a maccro with the prefix \texttt{POK\_NEEDS\_my-functionnality}
         \item
            When a maccro configures the kernel or user code, it has the prefix \\ \texttt{POK\_CONFIG\_my-config-directive}
      \end{enumerate}
   \item
      Indentation for ANY loop/condition
   \item
      Commits must be as small as possible
   \item
      Reduce machine-dependent code as more as possible
   \item
      Each header-file must begin with \texttt{\#ifdef \_\_POK\_SUBCATEGORY\_FILENAME\_H\_\_}
   \item
      Loop and condition style is :
      \begin{verbatim}
      condition 
      {
      }
      \end{verbatim}
   \textbf{and NOT}
      \begin{verbatim}
         condition {
         }
      \end{verbatim}
   \item
      If you introduce a new function for the userland, you must add relevant
      documentation in the \texttt{doc/userguide/} directory.
\end{enumerate}


\section{Source organization}
At the root directory, two main directories are available: \texttt{kernel} and
\texttt{libpok}. We detail the organization and the guidelines for each
subdirectory of kernel and libpok.

   \subsection{kernel}
   In the kernel, sources files are supposed to contain few lines of code. In
   consequence, there is one file for each service.
   \begin{itemize}
      \item
         \texttt{arch}: contains arch-dependent code. There is one directory for
         each architecture and one subdirectory for each BSP. For example, files
         for the x86 architecture and the x86-qemu BSP are located in the
         \texttt{arch/x86/x86-qemu} directory.
      \item
         \texttt{core}: contains the core functionnality of POK - threads,
         partitions, health monitoring, \ldots.
      \item
         \texttt{include}: contains headers files. The organization of header
         files is the same than source files. So, you will find \texttt{core},
         \texttt{middleware} or \texttt{arch} directories in the
         \texttt{include} directory.
      \item
         \texttt{libc}: provides some functionnalities for printing things.
         Functions located in this directory are here mainly for debugging
         purposes.
      \item
         \texttt{middleware}: contain the code for inter-partitions
         communication (sampling and queueing ports). It also contains some
         functionnalities about virtual ports routing.
   \end{itemize}

   \subsection{libpok}
   In libpok, sources files are supposed to contain more code than in kernel.
   So, there is one file for each functions. There is the organization and
   purpose of each directory.

   \begin{itemize}
      \item
         \texttt{arch}: contains architecture dependent files. Unlike the
         kernel, there is no need to separate each BSP so there is no
         subdirecties for each architecture.
      \item
         \texttt{arinc653}: contains the implementation of the ARINC653 layer.
      \item
         \texttt{core}: contains the main functionnality of POK. It contains the
         thread service, lockobjects, semaphores, events.
      \item
         \texttt{drivers}: contains device drivers implemented in POK.
      \item
         \texttt{include}: contains header files. As in the kernel, the
         structure of this directory follow the structure of the sources.
      \item
         \texttt{libc}: contains the C-library of POK (stdio, stdlib and so on).
      \item
         \texttt{libm}: contains the libmath backported from NetBSD.
      \item
         \texttt{middleware}: contains sources for sampling and queueing ports
         (interfacing with the kernel - inter-partition communication) but also blackboard and buffers
         (intra-partition communication)
   \end{itemize}


\section{Optimization (about the \texttt{POK\_CONFIG\_OPTIMIZE\_FOR\_GENERATED\_CODE})}
Systems generated with POK must be lightweight and keep a small memory footprint
to be compliant with embedded requirements and ensures a good code coverage.
When a system is written by hand, the libpok layer contains all its
functionnalities. It is more convenient for the developper, he does not have to
specify which functions he needs.

However, when a system is generated from AADL models, it defines the maccro 
\texttt{POK\_CONFIG\_OPTIMIZE\_FOR\_GENERATED\_CODE} and sets its values to 1.
Its means that the code specifies precisely which functions are used. Then, the
generated code specifies which services it needs using \texttt{POK\_NEEDS*}
maccros. For example, the \texttt{POK\_NEEDS\_LIBC\_STDIO} specifies that it
needs all functions of \texttt{libc/stdio}.

Then, each function of libpok is surrounded with a \texttt{POK\_CONFIG\_NEEDS\_FUNC*}
or \texttt{POK\_CONFIG\_NEEDS\_*}. {\LARGE You have to introduce that in your code when
you introduce new services in POK}.

Then, the file in \texttt{include/core/dependencies.h} specifies which functions
are needed for each service. When the
\texttt{POK\_CONFIG\_OPTIMIZE\_FOR\_GENERATED\_CODE} is not set, all functions
are enabled (default behavior). But is defined, functions are carefully
activated, depending on their service.


\section{Documentation}
   \subsection{User Guide}
   Each improvement and enhancement in kernel or libpok must be documented in the
   userguide (see \texttt{doc/userguide} in the POK sources) to keep a consistency
   between the documentation and the sources.

   \subsection{Code documentation}
   The code \textbf{must be} documented using doxygen. At each release, we issue
   a documentation in HTML and PDF using code documentation. The following
   paragraphs indicate at least what information should be included in the
   sources \textbf{at least}. Keep in mind that the more the code is documented,
   the best it is for users.

      \paragraph{Beginning of a file\\}
      Specify the file, the author, the data and a brief description. You can
      have an example in \texttt{kernel/core/thread.c}. For example, the
      following comments provide these informations. It should be located at the
      beginning of the file.
      \begin{verbatim}
/**
 * \file    core/thread.c
 * \author  Julien Delange
 * \date    2008-2009
 * \brief   Thread management in kernel
 */
      \end{verbatim}

      \paragraph{Functions\\}
      You \textbf{MUST} document each function and details what the function do.
      You specify that with a comment just before the function. The comment must
      begin with \texttt{/**}. There is an exemple for the function
      \texttt{pok\_thread\_init}:
      \begin{verbatim}
/**
 * Initialize threads array, put their default values
 * and so on
 */
void pok_thread_init(void)
{
...
      \end{verbatim}

      \paragraph{Global variables\\}
      Each global variable \textbf{must be} documented. As functions, you put a
      comment just before the global variable. This comment \textbf{must begin}
      with \texttt{/**}. There is an example for the global variable
      \texttt{pok\_threads}:
      \begin{verbatim}
/**
 * We declare an array of threads. The amount of threads
 * is fixed by the software developper and we add two theads
 *    - one for the kernel thread (this code)
 *    - one for the idle task
 */
pok_thread_t                     pok_threads[POK_CONFIG_NB_THREADS];
      \end{verbatim}


\section{Submit a patch}
If you found a bug or just want to send us an improvement, you can reach us at
the following address: \texttt{pok-devel at listes dot enst dot fr}. Please send
an email with the patch. We will answer and potentially merge your patch in the
current version of POK.

\section{Algorithms guidelines}
Before introducing new functions or modifying existing ones, please qualify your
code in terms of complexity, memory overhead, computation overhead, determinism.
POK targets safety-critical systems, and so, its functions must provide high
confidence to the user and must address these problems in its functions.

Moreover, we always follow the moto \textit{Keep It Simple, Stupid} for each
function: code must be understandable and documentation to be spread over users
or developpers.

\section{GDB'ing POK with QEMU}
POK allows you to attach a remote GDB to monitor the kernel \textit{or} its
partitions.

To do so, go to your example directory and run the system in debug mode.

\begin{verbatim}
$ cd $POK_PATH/examples/partition-threads
$ make run-gdb
\end{verbatim}

QEMU should be paused. Now run GDB using the kernel image.

\begin{verbatim}
$ gdb generated-code/kernel/kernel.elf
...
(gdb) target remote :1234
Remote debugging using :1234
0x0000fff0 in ?? ()
(gdb) continue
\end{verbatim}

You're all set if you want to debug the kernel, but what if you want to
instrumentate a partition?

In GDB, we first have to let the kernel know about the symbols of the
partition. But we also need to know where they are loaded in kernel space.
Let's say we want to debug partition \#1. One way to know where it was
relocated would be:

\begin{verbatim}
(gdb) p pok_partitions[0].base_addr
$1 = 1175552
\end{verbatim}

Please note that \texttt{pok\_partition\_init} \textit{must} have completed
or the array won't be initialized yet.

Now we can load the symbol table with the correct offset.

{
\normalsize

\begin{verbatim}
(gdb) add-symbol-file generated-code/cpu/part1/part1.elf 1175552
add symbol table from file "generated-code/cpu/part1/part1.elf" at
        .text_addr = 0x11f000
(y or n) y
Reading symbols from /home/laurent/pok/examples/partitions-threads/generated-code/cpu/part1/part1.elf...done.
(gdb) b user_hello_part1
Breakpoint 1 at 0x11f17a: file ../../../hello1.c, line 21.
(gdb) continue
Continuing.

Program received signal SIGTRAP, Trace/breakpoint trap.
0x0000017a in ?? ()
\end{verbatim}
}

You will notice debug symbols are missing, although we loaded them above.
This is because the memory mapping is not the same in kernel end userland.
We have to load the symbol file again in place of the kernel.

{
\normalsize

\begin{verbatim}
(gdb) symbol-file generated-code/cpu/part1/part1.elf
Load new symbol table from "/home/laurent/pok/examples/partitions-threads/generated-code/cpu/part1/part1.elf"? (y or n) y
Reading symbols from /home/laurent/pok/examples/partitions-threads/generated-code/cpu/part1/part1.elf...done.
(gdb) bt
#0  user_hello_part1 () at ../../../hello1.c:21
#1  0xc4830845 in ?? ()
\end{verbatim}
}

\subsection{Troubleshooting}
  \paragraph{GDB does not break on partition symbols} 
  \ \\
  In case your binaries are not generated by POK, it might happen, that GDB
  tells you it set a breakpoint at \texttt{user\_hello\_part1}, but it does not
  stop at this function.
  This happens, if the \texttt{.text} section of the binary file that you load 
  with \texttt{add-symbol-file} does not start at the beginning of the file.
  If \texttt{.text} has an offset of \texttt{0x1000} you need to add this
  offset to the address passed to \texttt{add-symbol-file} (see 
  \texttt{(gdb) help add-symbol-file} for details)

  {
    \normalsize
    \begin{verbatim}
    
    (gdb) p/x pok_partitions[0].base_addr
    $1 = 0x31c000
    (gdb) p/x pok_partitions[0].base_addr + 0x1000
    $2 = 0x31d000
    (gdb) add-symbol-file generated-code/cpu/part1/part1.elf 0x31d000
    ...
    \end{verbatim}
  }

\noindent Where exactly the \texttt{.text} section starts can be figured out with
  
{
  \normalsize
  \begin{verbatim}
$ readelf -S generated-code/cpu/part1/part1.elf
There are 22 section headers, starting at offset 0x2039b0:

Section Headers:
  [Nr] Name              Type            Addr     Off    Size   ES Flg Lk Inf Al
  [ 0]                   NULL            00000000 000000 000000 00      0   0  0
  [ 1] .text             PROGBITS        00001000 001000 01db18 00  AX  0   0 16
  [ 2] .init             PROGBITS        0001eb18 01eb18 00000d 00  AX  0   0  1
  [ 3] .fini             PROGBITS        0001eb25 01eb25 000008 00  AX  0   0  1
  [ 4] .rodata           PROGBITS        0001eb40 01eb40 00164c 00   A  0   0 32
  [ 5] .eh_frame         PROGBITS        0002018c 02018c 009df4 00   A  0   0  4
  [ 6] .ctors            PROGBITS        0002a000 02a000 000008 00  WA  0   0  4
  [ 7] .dtors            PROGBITS        0002a008 02a008 000008 00  WA  0   0  4
  [ 8] .jcr              PROGBITS        0002a010 02a010 000004 00  WA  0   0  4
  [ 9] .data             PROGBITS        0002a020 02a020 000a44 00  WA  0   0 32
  [10] .bss              NOBITS          0002aa80 02aa64 008ec0 00  WA  0   0 32
  [11] .comment          PROGBITS        00000000 02aa64 00008f 01  MS  0   0  1
  [12] .debug_aranges    PROGBITS        00000000 02aaf8 0028c0 00      0   0  8
  [13] .debug_info       PROGBITS        00000000 02d3b8 15f4bd 00      0   0  1
  [14] .debug_abbrev     PROGBITS        00000000 18c875 02702a 00      0   0  1
  [15] .debug_line       PROGBITS        00000000 1b389f 036d5e 00      0   0  1
  [16] .debug_str        PROGBITS        00000000 1ea5fd 011073 01  MS  0   0  1
  [17] .debug_loc        PROGBITS        00000000 1fb670 007c98 00      0   0  1
  [18] .debug_ranges     PROGBITS        00000000 203308 0005e8 00      0   0  1
  [19] .shstrtab         STRTAB          00000000 2038f0 0000bf 00      0   0  1
  [20] .symtab           SYMTAB          00000000 203d20 007520 10     21 1208  4
  [21] .strtab           STRTAB          00000000 20b240 006fad 00      0   0  1
Key to Flags:
  W (write), A (alloc), X (execute), M (merge), S (strings)
  I (info), L (link order), G (group), T (TLS), E (exclude), x (unknown)
  O (extra OS processing required) o (OS specific), p (processor specific)

  \end{verbatim}
}




\section{Commit}
Before pushing changes, please check that your test suite passes on all supported
architectures and that the document is up-to-date and can be properly built.

\section{Join the POK developper network !}
If you want to join the POK team, please send us an email (\texttt{pok-devel at
listes dot enst dot fr}. We are always looking for developpers with strong
skills in C, ASM and low-level programming.

If you are interested and think you can improve the project, you're welcome!


\end{document}
